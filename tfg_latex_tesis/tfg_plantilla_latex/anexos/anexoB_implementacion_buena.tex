\section{Implementación de Reglas de Alerta}
\label{sec:anexo_alertas}

\subsection{Evaluación de Caída de Actividad}
\label{subsec:anexo_actividad_drop}
La siguiente implementación muestra cómo se evalúa una caída significativa en la actividad física de un usuario. Esta regla compara los datos de los últimos 7 días con la actividad actual para detectar cambios significativos.

\begin{lstlisting}[language=Python, caption=Implementación de la regla de alerta para caída de actividad]
def check_activity_drop(user_id, current_date):
    """Verifica si hay una caída significativa en la actividad física."""
    try:
        # Obtener datos de los últimos 7 días
        start_date = current_date - timedelta(days=7)
        daily_summaries = get_daily_summaries(user_id, start_date, current_date)
        
        if not daily_summaries or len(daily_summaries) < 2:
            return False
            
        # Obtener el promedio de los últimos 7 días (excluyendo hoy)
        previous_days = [s for s in daily_summaries if s[2] < current_date.date()]
        if not previous_days:
            return False
            
        # Calcular promedios solo con valores no nulos y mayores que cero
        valid_steps = [s[3] for s in previous_days if s[3] is not None and s[3] > 0]
        valid_active_minutes = [s[9] for s in previous_days if s[9] is not None and s[9] > 0]
        
        if not valid_steps or not valid_active_minutes:
            return False
            
        avg_steps = sum(valid_steps) / len(valid_steps)
        avg_active_minutes = sum(valid_active_minutes) / len(valid_active_minutes)
        
        # Solo generar alertas si hay datos significativos
        if avg_steps < 100 or avg_active_minutes < 5:
            return False
            
        # Obtener los valores de hoy
        today_data = next((s for s in daily_summaries if s[2] == current_date.date()), None)
        if not today_data:
            return False
            
        today_steps = today_data[3] or 0
        today_active_minutes = today_data[9] or 0
            
        # Calcular porcentajes de caída
        steps_drop = ((avg_steps - today_steps) / avg_steps * 100)
        active_minutes_drop = ((avg_active_minutes - today_active_minutes) / avg_active_minutes * 100)
        
        db = DatabaseManager()
        if not db.connect():
            return False
            
        try:
            # Determinar la prioridad y el mensaje
            if steps_drop > 50 or active_minutes_drop > 50:
                priority = "high"
                threshold = 50.0
                drop_value = max(steps_drop, active_minutes_drop)
                details = (f"Caída severa en la actividad: {steps_drop:.1f}% menos pasos "
                          f"(de {avg_steps:.0f} a {today_steps}), "
                          f"{active_minutes_drop:.1f}% menos minutos activos "
                          f"(de {avg_active_minutes:.0f} a {today_active_minutes})")
                
                db.insert_alert(
                    user_id=user_id,
                    alert_type="activity_drop",
                    priority=priority,
                    triggering_value=drop_value,
                    threshold=threshold,
                    timestamp=current_date,
                    details=details
                )
                return True
                
            elif steps_drop > 30 or active_minutes_drop > 30:
                priority = "medium"
                threshold = 30.0
                drop_value = max(steps_drop, active_minutes_drop)
                details = (f"Caída moderada en la actividad: {steps_drop:.1f}% menos pasos "
                          f"(de {avg_steps:.0f} a {today_steps}), "
                          f"{active_minutes_drop:.1f}% menos minutos activos "
                          f"(de {avg_active_minutes:.0f} a {today_active_minutes})")
                
                db.insert_alert(
                    user_id=user_id,
                    alert_type="activity_drop",
                    priority=priority,
                    triggering_value=drop_value,
                    threshold=threshold,
                    timestamp=current_date,
                    details=details
                )
                return True
        finally:
            db.close()
            
    except Exception as e:
        print(f"Error al verificar caída de actividad: {e}")
    return False
\end{lstlisting}

Esta implementación muestra cómo se evalúa una caída significativa en la actividad física de un usuario. La función realiza las siguientes operaciones:

\begin{itemize}
    \item Obtiene los datos de los últimos 7 días del usuario
    \item Calcula los promedios de pasos y minutos activos
    \item Compara los valores actuales con los promedios históricos
    \item Genera alertas de alta prioridad si la caída es mayor al 50\%
    \item Genera alertas de prioridad media si la caída es mayor al 30\%
    \item Almacena las alertas en la base de datos con detalles específicos
\end{itemize}

Los umbrales utilizados (50\% y 30\%) están basados en estudios médicos que indican que una reducción mayor al 50\% se asocia con deterioro funcional acelerado en adultos mayores, mientras que una reducción mayor al 30\% ya es considerada significativa y merece atención.


\section{Implementación de Rutas Flask}
\label{sec:anexo_flask}

\subsection{Rutas Principales}
\label{subsec:anexo_rutas}
La siguiente implementación muestra las rutas más importantes del sistema, incluyendo la autenticación, el dashboard y las APIs para obtener datos.

\begin{lstlisting}[language=Python, caption=Implementación de rutas principales en Flask]
# Ruta de inicio de sesión
@app.route('/livelyageing/login', methods=['GET', 'POST'])
def login():
    if current_user.is_authenticated:
        return redirect(url_for('home'))

    if request.method == 'POST':
        username = request.form['username']
        password = request.form['password']
        if username == USERNAME and password == PASSWORD:
            user = User(username)
            login_user(user)
            return redirect(url_for('home'))
        else:
            flash('Usuario o contraseña incorrectos', 'danger')
    return render_template('login.html')

# Ruta principal del dashboard
@app.route('/livelyageing/home')
@login_required
def home():
    """
    Página principal del dashboard que muestra un resumen de la actividad
    de todos los usuarios y sus alertas más recientes.
    """
    db = DatabaseManager()
    if db.connect():
        try:
            # Obtener resumen diario más reciente de cada usuario
            daily_summaries = db.execute_query("""
                SELECT u.name, u.email, d.*
                FROM users u
                LEFT JOIN daily_summaries d ON u.id = d.user_id
                WHERE d.date = (SELECT MAX(date) FROM daily_summaries WHERE user_id = u.id)
                OR d.date IS NULL
                ORDER BY d.date DESC NULLS LAST
            """)
            
            # Obtener alertas no leídas
            unread_alerts = db.execute_query("""
                SELECT a.*, u.name as user_name
                FROM alerts a
                JOIN users u ON a.user_id = u.id
                WHERE a.is_read = FALSE
                ORDER BY a.timestamp DESC
                LIMIT 10
            """)
            
            return render_template('home.html',
                                daily_summaries=daily_summaries,
                                unread_alerts=unread_alerts)
        finally:
            db.close()
    return redirect(url_for('login'))

# API para obtener resumen diario
@app.route('/livelyageing/api/daily_summary')
@login_required
def get_daily_summary():
    """
    API que devuelve el resumen diario de actividad para todos los usuarios.
    """
    db = DatabaseManager()
    if db.connect():
        try:
            daily_summaries = db.execute_query("""
                SELECT u.name, u.email, d.*
                FROM users u
                LEFT JOIN daily_summaries d ON u.id = d.user_id
                WHERE d.date = (SELECT MAX(date) FROM daily_summaries WHERE user_id = u.id)
                OR d.date IS NULL
                ORDER BY d.date DESC NULLS LAST
            """)
            
            return jsonify({
                'status': 'success',
                'data': daily_summaries
            })
        finally:
            db.close()
    return jsonify({
        'status': 'error',
        'message': 'No se pudo conectar a la base de datos'
    }), 500

# API para obtener alertas
@app.route('/livelyageing/api/alerts')
@login_required
def get_user_alerts_api():
    """
    API que devuelve las alertas más recientes para todos los usuarios.
    """
    db = DatabaseManager()
    if db.connect():
        try:
            alerts = db.execute_query("""
                SELECT a.*, u.name as user_name
                FROM alerts a
                JOIN users u ON a.user_id = u.id
                ORDER BY a.timestamp DESC
                LIMIT 50
            """)
            
            return jsonify({
                'status': 'success',
                'data': alerts
            })
        finally:
            db.close()
    return jsonify({
        'status': 'error',
        'message': 'No se pudo conectar a la base de datos'
    }), 500
\end{lstlisting}

Esta implementación muestra las rutas más importantes del sistema:

\begin{itemize}
    \item \texttt{/livelyageing/login}: Maneja la autenticación de usuarios
    \item \texttt{/livelyageing/home}: Muestra el dashboard principal con resumen de actividad
    \item \texttt{/livelyageing/api/daily\_summary}: API para obtener datos de actividad
    \item \texttt{/livelyageing/api/alerts}: API para obtener alertas
\end{itemize}

Las características más importantes de esta implementación son:

\begin{itemize}
    \item Uso de decoradores \texttt{@login\_required} para proteger las rutas
    \item Manejo de sesiones y autenticación con Flask-Login
    \item Integración con la base de datos TimescaleDB
    \item Respuestas JSON para las APIs
    \item Consultas SQL optimizadas para obtener datos actualizados
\end{itemize}

\section{Implementación de Llamadas AJAX}
\label{sec:anexo_ajax}

\subsection{Manejo de Formularios y Actualizaciones}
\label{subsec:anexo_formularios}
La siguiente implementación muestra cómo se manejan las llamadas AJAX para la actualización dinámica de datos y el envío de formularios.

\begin{lstlisting}[language=JavaScript, caption=Implementación de llamadas AJAX para formularios y actualizaciones]
// Manejo del formulario de vinculación de dispositivo
document.addEventListener('DOMContentLoaded', function() {
    const linkForm = document.getElementById('linkForm');
    if (linkForm) {
        const submitBtn = linkForm.querySelector('#submitBtn');
        const spinner = submitBtn.querySelector('.spinner-border');
        const buttonText = submitBtn.querySelector('.button-text');

        linkForm.addEventListener('submit', function(e) {
            e.preventDefault();
            
            // Mostrar estado de carga
            submitBtn.disabled = true;
            spinner.classList.remove('d-none');
            buttonText.textContent = 'Procesando...';
            
            // Obtener datos del formulario
            const formData = new FormData(linkForm);
            
            // Enviar formulario usando fetch
            fetch(linkForm.action, {
                method: 'POST',
                body: formData,
                headers: {
                    'Accept': 'text/html',
                    'X-Requested-With': 'XMLHttpRequest'
                },
                credentials: 'same-origin'
            })
            .then(response => {
                if (response.redirected) {
                    window.location.href = response.url;
                } else {
                    return response.text().then(html => {
                        if (html.includes('assign_user')) {
                            window.location.href = '/livelyageing/assign';
                        } else {
                            document.documentElement.innerHTML = html;
                        }
                    });
                }
            })
            .catch(error => {
                console.error('Error:', error);
                alert('Error al enviar el formulario. Por favor, inténtalo de nuevo.');
            })
            .finally(() => {
                // Restaurar estado del botón
                submitBtn.disabled = false;
                spinner.classList.add('d-none');
                buttonText.textContent = 'Autorizar Fitbit';
            });
        });
    }
});

// Función auxiliar para añadir spinner a botones
function addSpinnerToButton(button) {
    if (!button.querySelector('.spinner-border')) {
        const spinner = document.createElement('span');
        spinner.className = 'spinner-border spinner-border-sm ms-2';
        spinner.setAttribute('role', 'status');
        spinner.setAttribute('aria-hidden', 'true');
        button.appendChild(spinner);
        button.disabled = true;
    }
}

// Añadir spinners a todos los botones de formulario
document.addEventListener('DOMContentLoaded', function() {
    const buttons = document.querySelectorAll('form button[type="submit"], form button[type="button"]');
    
    buttons.forEach(function(button) {
        if (button.type === 'button') {
            button.addEventListener('click', function(e) {
                if (!button.form || button.form.checkValidity()) {
                    addSpinnerToButton(button);
                }
            });
        }
        else if (button.type === 'submit') {
            button.form.addEventListener('submit', function(e) {
                if (button.form.checkValidity()) {
                    addSpinnerToButton(button);
                }
            });
        }
    });
});
\end{lstlisting}

Esta implementación muestra cómo se manejan las interacciones asíncronas en el sistema:

\begin{itemize}
    \item Uso de la API Fetch para realizar peticiones AJAX
    \item Manejo de estados de carga con spinners
    \item Procesamiento de respuestas y redirecciones
    \item Manejo de errores y recuperación
\end{itemize}

Las características más importantes de esta implementación son:

\begin{itemize}
    \item Feedback visual inmediato al usuario durante las operaciones
    \item Manejo de estados de carga y error
    \item Redirecciones automáticas basadas en la respuesta del servidor
    \item Reutilización de código para spinners en botones
    \item Validación de formularios antes del envío
\end{itemize}

\section{Implementación de Llamadas a la API de Fitbit}
\label{sec:anexo_fitbit}

\subsection{Obtención de Datos de Actividad y Frecuencia Cardíaca}
La implementación de las llamadas a la API de Fitbit se realiza a través de la función \texttt{get\_fitbit\_data}, que obtiene y procesa los datos de actividad física y frecuencia cardíaca del usuario. A continuación se muestra un fragmento relevante de la implementación:

\begin{lstlisting}[language=Python, caption=Implementación de la obtención de datos de Fitbit]
def get_fitbit_data(access_token, email):
    headers = {"Authorization": f"Bearer {access_token}"}
    today = datetime.now().strftime("%Y-%m-%d")
    user_id = get_latest_user_id_by_email(email)
    
    data = {
        'steps': 0,
        'distance': 0,
        'calories': 0,
        'floors': 0,
        'elevation': 0,
        'active_minutes': 0,
        'sedentary_minutes': 0,
        'heart_rate': 0,
        'sleep_minutes': 0,
        'nutrition_calories': 0,
        'water': 0,
        'spo2': 0,
        'respiratory_rate': 0,
        'temperature': 0
    }

    try:
        # Datos de actividad diaria
        activity_url = f"https://api.fitbit.com/1/user/-/activities/date/{today}.json"
        response = requests.get(activity_url, headers=headers)
        response.raise_for_status()
        
        activity_data = response.json()
        if 'summary' in activity_data:
            summary = activity_data['summary']
            data.update({
                'steps': summary.get('steps', 0),
                'distance': summary.get('distances', [{}])[0].get('distance', 0),
                'calories': summary.get('caloriesOut', 0),
                'floors': summary.get('floors', 0),
                'elevation': summary.get('elevation', 0),
                'active_minutes': summary.get('veryActiveMinutes', 0),
                'sedentary_minutes': summary.get('sedentaryMinutes', 0)
            })

        # Obtener frecuencia cardíaca en reposo
        resting_hr_url = f"https://api.fitbit.com/1/user/-/activities/heart/date/{today}/1d.json"
        response = requests.get(resting_hr_url, headers=headers)
        response.raise_for_status()
        
        hr_data = response.json()
        if 'activities-heart' in hr_data:
            for activity in hr_data['activities-heart']:
                if 'value' in activity and 'restingHeartRate' in activity['value']:
                    data['resting_heart_rate'] = activity['value']['restingHeartRate']
\end{lstlisting}

Esta implementación muestra cómo se obtienen los datos de actividad física y frecuencia cardíaca de la API de Fitbit. Las características más importantes son:

\begin{itemize}
    \item Uso de tokens de acceso para autenticación
    \item Manejo de errores con \texttt{raise\_for\_status()}
    \item Procesamiento de respuestas JSON con valores por defecto
    \item Obtención de múltiples tipos de datos en una sola función
\end{itemize}

\section{Implementación de Internacionalización}
\label{sec:anexo_i18n}

\subsection{Configuración y Uso de Flask-Babel}
La internacionalización se implementa utilizando Flask-Babel, que permite la traducción de textos en la aplicación. A continuación se muestra la configuración y ejemplos de uso:

\begin{lstlisting}[language=Python, caption=Configuración de Flask-Babel]
# Configuración de Flask-Babel
babel = Babel(app)

@babel.localeselector
def get_locale():
    # Primero intentar obtener el idioma de la sesión
    if 'language' in session:
        return session['language']
    # Luego intentar obtenerlo de las preferencias del navegador
    return request.accept_languages.best_match(['es', 'en'])
\end{lstlisting}

\begin{lstlisting}[language=HTML, caption=Ejemplo de uso en plantillas]
{{ _('User Dashboard') }} - Lively Ageing

<div class="alert alert-info">
    {{ _('Welcome back, %(name)s!', name=user.name) }}
</div>

<div class="stats-card">
    <h3>{{ _('Daily Activity') }}</h3>
    <p>{{ _('Steps: %(steps)d', steps=steps) }}</p>
    <p>{{ _('Active Minutes: %(minutes)d', minutes=active_minutes) }}</p>
</div>
\end{lstlisting}

Esta implementación muestra cómo se configura y utiliza la internacionalización en el sistema. Las características más importantes son:

\begin{itemize}
    \item Configuración de Flask-Babel con selector de idioma personalizado
    \item Soporte para múltiples idiomas (español e inglés)
    \item Uso de la función \texttt{\_()} para marcar textos traducibles
    \item Interpolación de variables en las traducciones
    \item Persistencia de la preferencia de idioma en la sesión
\end{itemize}
