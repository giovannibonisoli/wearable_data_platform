% Definición de colores y estilos para la tabla
\definecolor{headercolor}{RGB}{240,240,240}
\definecolor{rowcolor}{RGB}{250,250,250}
\label{anexo:anexo_umbrales}
% Configuración del espaciado y márgenes
\setlength{\tabcolsep}{8pt}

\begin{table}[htbp]
    \centering
    \renewcommand{\arraystretch}{1.4}
    \begin{tabularx}{\textwidth}{|>{\raggedright\arraybackslash}p{2.5cm}|>{\centering\arraybackslash}p{1.85cm}|>{\centering\arraybackslash}p{3.5cm}|X|}
        \hline
        \rowcolor{headercolor}
        \textbf{Tipo de Alerta} & \textbf{Prioridad} & \textbf{Umbral} & \textbf{Justificación y Referencia} \\
        \hline
        \multicolumn{4}{|l|}{\cellcolor{rowcolor}\textit{Actividad Física}} \\
        \hline
        Caída en actividad física & Alta & $>$50\% reducción & Asociado a deterioro funcional acelerado y mayor riesgo de hospitalización en mayores. \newline \textit{Smith et al., 2019} \\
        \hline
        Caída en actividad física & Media & $>$30\% reducción & Cambio significativo en el patrón habitual, permite intervención temprana. \newline \textit{Asociación Americana de Geriatría} \\
        \hline
        \multicolumn{4}{|l|}{\cellcolor{rowcolor}\textit{Sedentarismo}} \\
        \hline
        Aumento de tiempo sedentario & Alta & $>$50\% incremento & Relación dosis-respuesta con riesgo cardiovascular y metabólico. \newline \textit{Owen et al., 2020} \\
        \hline
        Aumento de tiempo sedentario & Media & $>$30\% incremento & Predice mayor riesgo de hospitalización y deterioro funcional. \newline \textit{Estudio LIFE} \\
        \hline
        \multicolumn{4}{|l|}{\cellcolor{rowcolor}\textit{Sueño}} \\
        \hline
        Cambio en duración del sueño & Alta & $>$30\% variación & Cambios de $\pm$30\% (2-2.5h) asociados a trastornos neurológicos y psiquiátricos. \newline \textit{Irwin, 2015} \\
        \hline
        \multicolumn{4}{|l|}{\cellcolor{rowcolor}\textit{Frecuencia Cardíaca}} \\
        \hline
        Anomalía en FC & Alta & $>$2 SD, $>$20\% lecturas anómalas & Patrón sostenido de irregularidad, riesgo de eventos cardiovasculares. \newline \textit{Chow et al., 2018} \\
        \hline
        Anomalía en FC & Media & $>$2 SD, $>$10\% lecturas anómalas & Detecta anomalías relevantes evitando falsas alarmas. \\
        \hline
        \multicolumn{4}{|l|}{\cellcolor{rowcolor}\textit{Validación de Datos}} \\
        \hline
        Rangos fisiológicos & - & \begin{tabular}[t]{l}
            Pasos: 0-50.000\\
            FC: 30-200 bpm\\
            Sueño: 0-1440 min\\
            SpO$_2$: 80-100\%
        \end{tabular} & Basado en límites fisiológicos y clínicos. Valores fuera de rango indican error de medición o situación crítica. \\
        \hline
        \multicolumn{4}{|l|}{\cellcolor{rowcolor}\textit{Inactividad}} \\
        \hline
        Inactividad intradía & Media/Alta & $\geq$2h sin pasos & Períodos prolongados de inactividad aumentan riesgo cardiovascular y de caídas. \newline \textit{Barone Gibbs et al., 2021} \\
        \hline
    \end{tabularx}
    \caption{Umbrales y criterios de alerta implementados en el sistema, con justificación clínica y referencias.}
    \label{tab:anexo_umbrales_alertas}
\end{table}
