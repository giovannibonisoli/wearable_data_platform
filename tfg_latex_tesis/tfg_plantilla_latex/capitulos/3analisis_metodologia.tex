% -*- coding: utf-8 -*-
\chapter{Análisis y Metodología}
\label{chap:requisitos_metodologia}

Este capítulo detalla los requisitos funcionales y no funcionales del sistema, así como la metodología seguida durante su desarrollo.

\section{Requisitos del Sistema}
\label{sec:requisitos}

\subsection{Requisitos Funcionales}
\label{subsec:requisitos_funcionales}

Los requisitos funcionales definen las capacidades específicas que el sistema debe proporcionar. La Tabla~\ref{tab:requisitos_funcionales} resume los requisitos identificados, agrupados por área funcional y con su prioridad asociada.

\begin{table}[htbp]
\caption{Requisitos Funcionales del Sistema}
\label{tab:requisitos_funcionales}
\begin{tabular}{|p{0.17\textwidth}|p{0.65\textwidth}|p{0.12\textwidth}|}
\hline
\textbf{ID} & \textbf{Descripción} & \textbf{Prioridad} \\
\hline
\multicolumn{3}{|l|}{\textit{Gestión de Usuarios y Autenticación}} \\
\hline
RF-01 & Autenticación del personal mediante credenciales compartidas y gestión de sesiones & Alta \\
RF-02 & Vinculación y gestión de cuentas Fitbit (OAuth 2.0) con nombres identificativos & Alta \\
\hline
\multicolumn{3}{|l|}{\textit{Adquisición y Almacenamiento}} \\
\hline
RF-03 & Obtención automática y periódica de datos de salud y actividad vía API Fitbit & Alta \\
RF-04 & Persistencia estructurada de datos en TimescaleDB con asociación usuario-tiempo & Alta \\
\hline
\multicolumn{3}{|l|}{\textit{Procesamiento y Alertas}} \\
\hline
RF-05 & Evaluación continua de datos contra umbrales predefinidos y personalizados & Alta \\
RF-06 & Generación y registro de alertas con priorización y contextualización & Alta \\
\hline
\multicolumn{3}{|l|}{\textit{Visualización y Gestión}} \\
\hline
RF-07 & Dashboard web para visualización de datos históricos y alertas & Media \\
RF-08 & Gestión de reasignación de dispositivos y actualización de vinculaciones & Media \\
\hline
\end{tabular}
\end{table}

\subsection{Requisitos No Funcionales}
\label{subsec:requisitos_no_funcionales}

La Tabla~\ref{tab:requisitos_no_funcionales} presenta los requisitos no funcionales, agrupados por categorías de atributos de calidad.
\begin{table}[htbp]
\caption{Requisitos No Funcionales del Sistema}
\label{tab:requisitos_no_funcionales}
\begin{tabular}{|p{0.17\textwidth}|p{0.75\textwidth}|}
\hline
\textbf{Categoría} & \textbf{Requisitos} \\
\hline
Seguridad & 
\begin{itemize}
\item Autenticación robusta y gestión segura de sesiones
\item Cifrado de tokens OAuth 2.0 y comunicaciones HTTPS
\item Protección contra vulnerabilidades web (OWASP Top 10)
\end{itemize} \\
\hline
Rendimiento & 
\begin{itemize}
\item Adquisición eficiente respetando límites de API
\item Consultas optimizadas en TimescaleDB
\item Dashboard web con carga fluida
\end{itemize} \\
\hline
Fiabilidad & 
\begin{itemize}
\item Ejecución robusta de scripts programados
\item Manejo adecuado de errores y registro de incidencias
\item Minimización de falsos positivos en alertas
\end{itemize} \\
\hline
Usabilidad & 
\begin{itemize}
\item Interfaz intuitiva para personal autorizado
\item Mensajes claros y orientativos
\item Visualización efectiva de datos y alertas
\end{itemize} \\
\hline
Mantenibilidad & 
\begin{itemize}
\item Código modular y documentado
\item Control de versiones (Git)
\item Arquitectura extensible
\end{itemize} \\
\hline
\end{tabular}
\end{table}

Los detalles técnicos de implementación de estos requisitos se desarrollan en los capítulos posteriores.

\section{Metodología de Desarrollo}
\label{sec:metodologia}

El desarrollo siguió un enfoque iterativo e incremental adaptado a la naturaleza exploratoria del proyecto, con las siguientes fases principales:

\begin{enumerate}
    \item \textbf{Investigación y Definición (Fase Inicial):}
        \begin{itemize}
            \item Revisión bibliográfica sobre monitorización remota, wearables y tecnologías relevantes.
            \item Estudio detallado de la documentación de la API de Fitbit\textsuperscript{\textregistered} y el protocolo OAuth 2.0 con PKCE.
            \item Definición inicial de los objetivos y alcance del proyecto en colaboración con el tutor.
            \item Identificación de los requisitos funcionales y no funcionales preliminares.
        \end{itemize}
    \item \textbf{Diseño de la Arquitectura y Tecnologías:}
        \begin{itemize}
            \item Toma de decisiones sobre la arquitectura general: aplicación web Flask, scripts Python independientes para adquisición de datos, base de datos PostgreSQL con TimescaleDB para métricas, y programación de tareas periódicas mediante \textbf{cron}.
            \item Selección de las tecnologías principales (Python, Flask, psycopg2, cryptography, etc.).
            \item Diseño del esquema de la base de datos (tabla `users` y estructura pensada para tablas de métricas) y las interfaces entre componentes (rutas Flask, funciones en módulos Python).
        \end{itemize}
    \item \textbf{Implementación Iterativa (Ciclos de Desarrollo):}
        \begin{itemize}
            \item Desarrollo incremental de la funcionalidad principal, priorizando los módulos clave:
            \item Implementación del flujo de autenticación OAuth 2.0 con Fitbit\textsuperscript{\textregistered} (`auth.py`).
            \item Desarrollo del módulo de base de datos (`db.py`) incluyendo cifrado de tokens (`encryption.py`).
            \item Creación de la aplicación web Flask (`app.py`) con las rutas para la gestión de vinculaciones y autenticación del personal.
            \item Desarrollo de los scripts independientes para la adquisición de datos diarios (`fitbit.py`) e intradía (`fitbit\_intraday.py`).
            \item Configuración de la ejecución programada mediante \textbf{cron} y scripts `.sh`.
            \item Tratamiento de los datos y generación de alertas.
            \item Implementación de la interfaz de visualización (dashboard web con Flask y plantillas HTML/JS).
            \item Realización de pruebas funcionales manuales y depuración durante el desarrollo.
        \end{itemize}
    \item \textbf{Pruebas y Validación:}
        \begin{itemize}
            \item Ejecución de pruebas sobre el prototipo desplegado (en VM) para verificar el cumplimiento de los requisitos implementados (vinculación, adquisición, almacenamiento básico).
            \item Pruebas del flujo completo de vinculación y adquisición programada.
            \item Depuración y corrección de errores encontrados.
        \end{itemize}
    \item \textbf{Documentación:}
        \begin{itemize}
            \item Redacción de la memoria del TFG (este documento).
            \item Comentarios en el código fuente.
            \item Elaboración de diagramas y esquemas necesarios.
        \end{itemize}
\end{enumerate}

Para la gestión del código fuente y el control de versiones se utilizó \textbf{Git}, alojando el repositorio centralizado en la plataforma \textbf{GitHub} \cite{github_repo_proyecto}, lo que permitió un seguimiento detallado de los cambios y la posibilidad de colaboración. La gestión de tareas se realizó mediante seguimiento personal y comunicación con el tutor.

\section{Herramientas de Apoyo a la Redacción}
\label{sec:apoyo_redaccion}

La redacción de este TFG ha requerido un esfuerzo significativo para garantizar la claridad, coherencia y precisión técnica. Para optimizar el proceso, se empleó una estrategia mixta: la generación inicial de contenido se realizó mediante dictado por voz y escritura directa en procesadores de texto, priorizando la captura ágil de ideas y descripciones técnicas.

Posteriormente, se utilizó asistencia de modelos de lenguaje avanzados, en particular Gemini (Google), como herramienta de apoyo para:
\begin{itemize}
    \item Mejorar la claridad y concisión de las frases.
    \item Corregir errores gramaticales y de estilo.
    \item Mantener un tono formal y académico homogéneo.
    \item Optimizar la estructura y el flujo de los párrafos.
\end{itemize}
El uso de Gemini se limitó a la reformulación y revisión lingüística, bajo directrices precisas, sin delegar la responsabilidad sobre el contenido técnico, la estructura ni la validación final, que son íntegramente del autor. Esta asistencia permitió agilizar el pulido del texto y elevar la calidad formal de la memoria, sin comprometer el rigor ni la autoría intelectual.
