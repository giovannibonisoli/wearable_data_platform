\chapter{Resultados y Evaluación}

\section{Resultados de las Pruebas}

\subsection{Metodología de Evaluación}
Las pruebas se realizaron en un entorno controlado utilizando datos sintéticos generados específicamente para validar cada aspecto del sistema. La metodología incluyó:

\begin{itemize}
    \item \textbf{Generación de Datos}: Scripts Python personalizados para crear series temporales con patrones normales y anómalos.
    \item \textbf{Medición de Métricas}: Logging automático de resultados mediante el módulo \texttt{logging} de Python.
    \item \textbf{Validación}: Comparación automática de alertas generadas contra casos esperados predefinidos.
\end{itemize}

Los detalles completos de la metodología, incluyendo el entorno de pruebas y los datos sintéticos generados, se encuentran en el Anexo~\ref{anexo:pruebas}, Secciones~\ref{anexo:pruebas:entorno} y~\ref{anexo:pruebas:insertados}.

\subsection{Pruebas de Calidad de Datos}
Las pruebas de calidad de datos se realizaron exhaustivamente para validar la robustez del sistema. La Tabla~\ref{tab:resultados_calidad} presenta las métricas agregadas:

\begin{table}[H]
\centering
\caption{Resultados agregados de las pruebas de calidad de datos}
\label{tab:resultados_calidad}
\begin{tabular}{|l|c|c|}
\hline
\textbf{Métrica} & \textbf{Valor} & \textbf{Interpretación} \\ \hline
Tasa de Detección (Recall) & 100\% & Detección de todas las anomalías críticas \\
Precisión & 100\% & Sin falsos positivos en casos críticos \\
Falsos Positivos & 0 & No se generaron alertas incorrectas \\
Falsos Negativos & 3 & Solo en cambios moderados de sueño \\
\hline
\end{tabular}
\end{table}

El sistema demostró una capacidad robusta para detectar:
\begin{itemize}
    \item Datos faltantes críticos (pasos, frecuencia cardíaca, sueño)
    \item Valores anómalos fuera de rangos fisiológicos (SpO$_2$ < 80\%, FC > 200 bpm)
    \item Inconsistencias en los datos parciales
    \item Patrones anómalos en la frecuencia cardíaca (desviación > 2 SD)
    \item Cambios significativos en la actividad física (>30\% reducción)
\end{itemize}

Los logs detallados de estas pruebas, incluyendo los escenarios evaluados y resultados específicos, se encuentran en el Anexo~\ref{anexo:pruebas}, Sección~\ref{anexo:pruebas:calidad_datos}.

\subsection{Pruebas de Umbrales}
Las pruebas de umbrales validaron la efectividad del sistema en la detección de cambios significativos. Los resultados mostraron que:

\begin{itemize}
    \item El sistema detectó correctamente todas las alertas de alta prioridad (>50\% reducción en actividad o >2 SD en FC)
    \item Las alertas de prioridad media (>30\% reducción en actividad o >1.5 SD en FC) se generaron según lo esperado
    \item Se mantuvo un balance adecuado entre sensibilidad y especificidad
\end{itemize}

La validación detallada de los umbrales, incluyendo los escenarios específicos evaluados y los resultados de cada prueba, se documentan en el Anexo~\ref{anexo:pruebas}, Sección~\ref{anexo:pruebas:umbrales}.

\subsection{Pruebas Combinadas}
Las pruebas combinadas evaluaron la capacidad del sistema para manejar múltiples condiciones simultáneamente. Los resultados mostraron que:

\begin{itemize}
    \item El sistema es capaz de generar y manejar múltiples alertas simultáneamente
    \item Las alertas se generan de forma independiente y no interfieren entre sí
    \item La priorización de alertas funciona correctamente según la severidad de cada condición
\end{itemize}

Los logs de las pruebas combinadas, incluyendo los escenarios evaluados y el análisis detallado de los resultados, se encuentran en el Anexo~\ref{anexo:pruebas}, Sección~\ref{anexo:pruebas:combinadas}.

\section{Discusión Crítica y Limitaciones}

\subsection{Análisis de Falsos Negativos en Sueño}
Los 3 falsos negativos en la detección de cambios de sueño se analizaron en detalle:
\begin{itemize}
    \item \textbf{Causa}: Los umbrales actuales (30\%) son demasiado estrictos para cambios moderados
    \item \textbf{Impacto}: Cambios del 28.5\% no se detectan, aunque son clínicamente relevantes
    \item \textbf{Solución Propuesta}: Implementar umbrales adaptativos basados en la variabilidad histórica del usuario
\end{itemize}

El análisis detallado de estos casos y los logs correspondientes se encuentran en el Anexo~\ref{anexo:pruebas}, Sección~\ref{anexo:pruebas:umbrales}, donde se documentan los escenarios específicos y los resultados obtenidos.

\subsection{Limitaciones del Sistema}
\begin{itemize}
    \item \textbf{Entorno de Pruebas}: Las pruebas se realizaron con datos sintéticos, lo que limita la validación en condiciones reales
    \item \textbf{Escala}: No se han realizado pruebas con un gran número de usuarios simultáneos
    \item \textbf{Complejidad}: El sistema actual implementa reglas básicas de alerta, sin modelos predictivos avanzados
    \item \textbf{Interferencias}: No se han probado escenarios de interferencia de dispositivos o problemas de conexión
\end{itemize}

\subsection{Comparación con Otros Enfoques}
Comparado con otros sistemas de RPM basados en Fitbit:
\begin{itemize}
    \item \textbf{Ventajas}:
    \begin{itemize}
        \item Detección simultánea de múltiples anomalías
        \item Personalización de umbrales por usuario
        \item Procesamiento en tiempo real
    \end{itemize}
    
    \item \textbf{Áreas de Mejora}:
    \begin{itemize}
        \item Implementación de modelos predictivos
        \item Integración con otros dispositivos médicos
        \item Análisis de tendencias a largo plazo
    \end{itemize}
\end{itemize}

\section{Conclusiones}
Los resultados demuestran que el sistema cumple con los objetivos establecidos, aunque con limitaciones importantes:

\begin{itemize}
    \item \textbf{Precisión}: Alta precisión en la detección de anomalías críticas (100\%), pero con margen de mejora en cambios moderados (83.3\%)
    \item \textbf{Robustez}: Manejo efectivo de diferentes escenarios, pero con limitaciones en cambios graduales
    \item \textbf{Validación}: Necesidad de pruebas piloto en entorno real antes de implementación completa
\end{itemize}

Las áreas de mejora identificadas, como la detección de cambios en la duración del sueño y la validación en entornos reales, serán abordadas en futuras iteraciones del sistema. Los resultados obtenidos validan la efectividad del sistema para su uso en entornos controlados de monitorización de salud, aunque se recomienda una fase de pruebas piloto en un entorno real antes de su implementación completa.

Los logs completos de todas las pruebas, incluyendo los datos sintéticos utilizados y los resultados detallados, se encuentran disponibles en el Anexo~\ref{anexo:pruebas}.