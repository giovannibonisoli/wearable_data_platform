% -*- coding: utf-8 -*-
\chapter{Estado del Arte y Marco Tecnológico}
\label{chap:estado_arte}

Este capítulo revisa el estado actual de la monitorización remota de salud en personas mayores, el papel de los dispositivos vestibles como Fitbit\textsuperscript{\textregistered}, y describe las tecnologías clave empleadas en el sistema desarrollado, justificando su elección.

\section{Monitorización Remota de Salud en Personas Mayores}
\label{sec:ea_monitorizacion_remota}

La monitorización remota de pacientes (RPM, por sus siglas en inglés, \textbf{Remote Patient Monitoring}) ha ganado relevancia impulsada por la necesidad de modelos de atención sanitaria más eficientes, especialmente para la población mayor \cite{noah2022mobile}. En España, donde más del 20\% de la población supera los 65 años \cite{ine_proyeccion_2022_2072}, la RPM permite la detección temprana de deterioros y la mejora de la independencia de usuarios y cuidadores \cite{bashshur2018telemedicine}.

Las aproximaciones a la RPM varían desde sensores ambientales hasta dispositivos médicos específicos o wearables de consumo \cite{majumder2017wearable}. Los principales desafíos incluyen la usabilidad para usuarios mayores, la gestión de datos masivos, la fiabilidad de las mediciones y la privacidad de datos sensibles \cite{lee2021challenges}.

\subsection{Bases científicas y enfoque práctico para la detección de alertas}
\label{subsec:bases_alertas}

El sistema implementa reglas y umbrales basados en evidencia científica, adaptados a las limitaciones de los datos disponibles por Fitbit. Los principales parámetros monitorizados incluyen actividad física (caídas del 30-50\% respecto a la media semanal) \cite{rebelo_physical_inactivity_consequences_2020, who_guidelines_2020}, sedentarismo \cite{bellettiere_pa_sedentary_aging_women_2017}, patrones de sueño \cite{nsf_older_adult_sleep_2022} y frecuencia cardíaca \cite{kang_hrv_thresholds_mortality_2021}. Los umbrales detallados y su justificación se documentan en el Anexo~\ref{anexo:validacion_alertas}.

\subsection{Sistemas de alerta en salud digital}
\label{subsec:sistemas_alerta_comparativa}

La generación de alertas en este sistema se basa en un enfoque de reglas y umbrales explícitos, implementados en el backend (véase \texttt{alert\_rules.py}), por ser la opción más transparente y trazable para entornos clínicos y de cuidado. Este método, ampliamente utilizado en salud digital \cite{alam_alert_systems_review_2019}, permite adaptar fácilmente los criterios a nuevas evidencias o necesidades del usuario. Aunque existen enfoques más complejos (como modelos predictivos o integración de múltiples fuentes), en este TFG se prioriza la robustez, la interpretabilidad y la facilidad de validación.

Las alertas generadas se notifican a los cuidadores a través del panel web, donde pueden consultarse, filtrar por prioridad y marcar como revisadas.

\subsubsection{Efectividad y limitaciones del enfoque}
La efectividad de los sistemas de alerta depende tanto de la calidad de los datos como de la calibración de los umbrales. El sistema implementado busca minimizar la fatiga de alertas mediante la personalización de los umbrales y la priorización de alertas relevantes, pero reconoce limitaciones inherentes: posibles falsas alarmas si los patrones individuales varían mucho, y la imposibilidad de detectar eventos no reflejados en los datos de Fitbit. La validación empírica y la revisión periódica de los umbrales son esenciales para mantener la utilidad clínica del sistema.

\subsubsection{Justificación y validación de umbrales}
La definición de umbrales se basa en la literatura científica y en la experiencia clínica, pero se adapta a la variabilidad interindividual mediante el uso de porcentajes y comparación con la línea base personal. La justificación detallada de cada umbral y ventana temporal utilizada se documenta en los archivos técnicos del proyecto y se aborda en los capítulos de metodología e implementación.

\section{Dispositivos Wearables: El Caso de Fitbit}
\label{sec:ea_fitbit}

El mercado de dispositivos wearables ha experimentado un crecimiento exponencial \cite{fortune_wearable_market}. Fitbit\textsuperscript{\textregistered}, ahora parte de Google, es líder en pulseras y relojes de actividad física, con dispositivos que incluyen acelerómetros y fotopletismógrafos (PPG) para medir movimiento y frecuencia cardíaca \cite{fitbit_how_hr_works}.

Su API proporciona datos diarios e intradía de actividad física, sueño y frecuencia cardíaca \cite{fitbit_api_reference}. Aunque no son dispositivos médicos certificados, estudios validan su precisión aceptable en mediciones como frecuencia cardíaca en reposo, fases del sueño y conteo de pasos, con ciertas limitaciones según el dispositivo y condiciones de uso \cite{haghayegh2019accuracy, nelson2016validity}. Estas consideraciones son cruciales al interpretar los datos. La adquisición por Google podría afectar la disponibilidad futura de la API \cite{google_fitbit_acquisition_info}.
\section{Tecnologías Habilitadoras}
\label{sec:ea_tecnologias}

\subsection{API de Fitbit y OAuth 2.0}
\label{subsec:ea_fitbit_api_oauth}
El acceso a los datos de los usuarios de Fitbit\textsuperscript{\textregistered} se realiza exclusivamente a través de su API web oficial. Se trata de una API RESTful que utiliza el formato JSON para el intercambio de datos \cite{fitbit_api_reference}. Proporciona diversos \textit{endpoints} para obtener información del perfil del usuario, resúmenes de actividad diaria, datos de series temporales (como frecuencia cardíaca o pasos a lo largo del día con cierta granularidad), información sobre el sueño, etc. Para poder acceder a los datos de un usuario, es imprescindible obtener su consentimiento explícito a través del protocolo de autorización estándar \textbf{OAuth 2.0} \cite{oauth_spec_rfc6749}.

En este proyecto, se implementa el flujo \textit{Authorization Code Grant} de OAuth 2.0, considerado el más seguro para aplicaciones web con backend. Los detalles técnicos completos del proceso de autenticación y autorización, incluyendo diagramas de secuencia, ejemplos de respuestas de la API, y la gestión de tokens, se encuentran documentados en el Anexo~\ref{anexo:oauth_fitbit}.
La correcta y segura gestión de estos tokens (almacenamiento cifrado o seguro, uso exclusivo en el backend, uso de HTTPS en todas las comunicaciones) es fundamental para la seguridad y privacidad del sistema \cite{oauth_security_bcp_rfc8252}.

\subsection{Arquitecturas de Microservicios}
\label{subsec:ea_microservicios}

Frente a las arquitecturas tradicionales, donde toda la funcionalidad de la aplicación reside en un único proceso desplegable, la arquitectura de microservicios estructura la aplicación como una colección de servicios pequeños, autónomos y débilmente acoplados \cite{fowler_microservices}. Cada servicio se centra en una capacidad de negocio específica, se comunica con otros servicios a través de APIs bien definidas (normalmente sobre HTTP/REST o colas de mensajes) y puede ser desarrollado, desplegado y escalado de forma independiente \cite{newman_building_microservices}.

Las ventajas clave de este enfoque, relevantes para nuestro sistema, incluyen:
\begin{itemize}
    \item \textbf{Escalabilidad Independiente:} Cada servicio puede escalarse horizontalmente según sus necesidades específicas (ej. escalar más instancias del servicio de adquisición de datos si hay muchos usuarios).
    \item \textbf{Flexibilidad Tecnológica:} Cada servicio puede desarrollarse con la tecnología más adecuada para su tarea específica (diferentes lenguajes, bases de datos).
    \item \textbf{Despliegue Independiente:} Los cambios en un servicio pueden desplegarse sin necesidad de redesplegar todo el sistema, agilizando las actualizaciones.
\end{itemize}

\subsection{Bases de Datos de Series Temporales}
\label{subsec:ea_db_timeseries}

Los datos generados por dispositivos wearables como Fitbit\textsuperscript{\textregistered} son inherentemente datos de series temporales: secuencias de mediciones indexadas por tiempo (timestamp). Si bien es posible almacenar estos datos en bases de datos relacionales tradicionales (como PostgreSQL o MySQL), las bases de datos especializadas en series temporales (TSDB - Time Series Databases) están optimizadas para este tipo de carga de trabajo \cite{dbengines_timeseries_ranking}.

Las TSDB suelen ofrecer ventajas significativas para datos de series temporales, como:
\begin{itemize}
    \item \textbf{Alto Rendimiento en Ingesta:} Optimizadas para escribir grandes volúmenes de datos nuevos secuencialmente en el tiempo.
    \item \textbf{Consultas Eficientes Basadas en Tiempo:} Indexación y funciones específicas para agregar, muestrear o filtrar datos por rangos de tiempo de forma muy rápida.
    \item \textbf{Compresión de Datos:} Técnicas específicas para comprimir datos temporales, que suelen tener cierta redundancia o patrones, ahorrando espacio de almacenamiento.
    \item \textbf{Políticas de Retención de Datos:} Facilidades para descartar automáticamente datos antiguos que ya no son necesarios (ej. mantener datos con granularidad de minutos por 1 mes, pero solo resúmenes diarios después de eso).
\end{itemize}
Ejemplos populares de TSDB incluyen InfluxDB y TimescaleDB (una extensión para PostgreSQL) \cite{influxdb_docs, timescaledb_docs}. Para este proyecto, se optó por \textbf{TimescaleDB} debido a su integración nativa con PostgreSQL, lo que permite combinar las ventajas de una TSDB con las capacidades de una base de datos relacional robusta, su uso de SQL estándar para las consultas y su madurez como proyecto \cite{timescaledb_docs}.

\subsection{Herramientas de Backend y Procesamiento}
\label{subsec:ea_backend_tools}

El backend del sistema, responsable de orquestar la autenticación, la adquisición de datos, el procesamiento y la exposición de APIs internas o para el frontend, se ha desarrollado utilizando \textbf{Python}. Python es una elección popular para el desarrollo web y el procesamiento de datos debido a su sintaxis clara, su amplio ecosistema de librerías y su gran comunidad \cite{python_website}.

Como framework web para construir las APIs de los microservicios, se ha empleado \textbf{Flask} \cite{flask_docs}, un microframework ligero que se alinea con la filosofía de microservicios al mantener cada componente simple y modular.

Para la adquisición periódica de datos desde la API de Fitbit\textsuperscript{\textregistered} (una tarea que debe ejecutarse de forma programada en segundo plano para cada usuario vinculado), se utiliza \textbf{crontab}, el planificador de tareas estándar de sistemas Unix/Linux. Crontab permite definir trabajos (jobs) que se ejecutan según expresiones cron, siendo una solución robusta y probada para la ejecución programada de scripts.

\subsection{Tecnologías de Frontend/Visualización}
\label{subsec:ea_frontend_viz}

Para presentar la información monitorizada de forma clara y útil a los cuidadores, se ha desarrollado un panel web sencillo e intuitivo. Dada la base tecnológica en Python/Flask y la naturaleza de los datos (series temporales, gráficos estadísticos), la interfaz se construye con plantillas HTML, CSS y JavaScript integradas en Flask. La visualización se realiza mediante librerías JavaScript como Chart.js, permitiendo mostrar líneas de tiempo, resúmenes e indicadores clave de manera interactiva y comprensible. Esta solución prioriza la simplicidad y la mantenibilidad.

\section{Consideraciones Éticas y Legales (RGPD)}
\label{sec:ea_rgpd}

El sistema implementa los principios fundamentales del RGPD: consentimiento explícito mediante OAuth 2.0, uso limitado y minimizado de los datos, exactitud y conservación adecuada, y medidas técnicas y organizativas para garantizar la seguridad y confidencialidad (como HTTPS, almacenamiento seguro de credenciales y control de accesos). Se facilita el ejercicio de los derechos de los usuarios (acceso, rectificación, supresión, etc.) mediante mecanismos accesibles en la propia aplicación. La documentación de políticas y registros de consentimiento permite demostrar el cumplimiento normativo.