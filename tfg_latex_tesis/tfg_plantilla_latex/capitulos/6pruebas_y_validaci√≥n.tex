% -*- coding: utf-8 -*-
\chapter{Pruebas y Validación}
\label{chap:pruebas_validacion}

Este capítulo presenta las pruebas realizadas para validar el sistema de alertas médicas, su robustez ante datos anómalos y la precisión de la lógica implementada. Se evaluó tanto el comportamiento del sistema ante condiciones controladas como su fiabilidad en la detección simultánea de múltiples anomalías clínicas.

\section{Metodología de Validación}
\label{sec:metodologia_validacion}

Las pruebas se desarrollaron en un entorno aislado con una base de datos TimescaleDB dedicada, asegurando control total sobre los datos y reproducibilidad. Se utilizaron scripts automatizados en Python para:

\begin{itemize}
    \item Insertar datos sintéticos normales y anómalos para varios usuarios.
    \item Ejecutar las reglas de alerta del sistema.
    \item Registrar y verificar las alertas generadas.
\end{itemize}

\section{Pruebas Realizadas}
\label{sec:pruebas_realizadas}

\subsection{Test de Validación de Umbrales}
\label{subsec:test_thresholds}

Se diseñó un conjunto de datos sintéticos representando 20 días de actividad normal seguidos de 3 días con datos anómalos. Para cada uno de los tres usuarios de prueba, se introdujeron valores que debían activar alertas específicas (actividad física, sueño, sedentarismo, frecuencia cardíaca y calidad de datos).  
El sistema generó las alertas esperadas correctamente en el 90\% de los casos. La única excepción fue la alerta por disminución del sueño, con una reducción del 28,5\%, que no alcanzó el umbral del 30\% definido para su activación.

\vspace{1em}
\noindent\textbf{Resumen de resultados:}
\begin{itemize}
    \item \textbf{Alertas generadas correctamente:} 16 de 18
    \item \textbf{Tipos de alerta evaluados:} Actividad, Sedentarismo, Sueño, Frecuencia cardíaca, Calidad de datos, Inactividad intradía
    \item \textbf{Precisión:} 88.9\%
\end{itemize}

\subsection{Test de Inserción Controlada}
\label{subsec:test_insertions}

Complementando el test anterior, se validó el comportamiento del sistema ante inserciones intencionadas con:
\begin{itemize}
    \item \textbf{Datos normales}, para verificar que no se generan falsos positivos.
    \item \textbf{Anomalías específicas}, diseñadas para superar los umbrales críticos.
\end{itemize}

Se confirmaron los siguientes aspectos:
\begin{itemize}
    \item Coherencia en la activación de alertas por día y por tipo.
    \item Ausencia de alertas en datos normales.
    \item Prioridades correctamente asignadas (alta, media).
\end{itemize}

\begin{table}[H]
\centering
\caption{Resumen de precisión del sistema de alertas}
\begin{tabular}{|l|c|}
\hline
\textbf{Métrica} & \textbf{Valor} \\ \hline
Alertas totales evaluadas & 36 \\
Alertas esperadas y disparadas & 18 \\
Falsos positivos & 0 \\
Falsos negativos & 3 \\
Tasa de detección (Recall) & 83.3\% \\
Precisión (Precision) & 100.0\% \\
Cobertura de escenarios clínicos & 100\% \\ \hline
\end{tabular}
\label{tab:resumen_alertas}
\end{table}
La cobertura de escenarios clínicos se refiere al hecho de que todas las categorías de alerta (actividad, sedentarismo, sueño, frecuencia cardíaca, inactividad intradía y calidad de datos) fueron correctamente evaluadas al menos una vez, demostrando la completitud funcional del sistema.

Como se observa en la Tabla \ref{tab:resumen_alertas}, el sistema mostró un comportamiento robusto, con una precisión del 100\% y una tasa de detección del 83,3\%, penalizada únicamente por 3 falsos negativos en la alerta de duración del sueño. Este resultado respalda la consistencia del motor de reglas bajo condiciones simuladas controladas

\subsection{Test de Calidad de Datos}
\label{subsec:test_data_quality}

Se diseñó un conjunto específico de pruebas para validar la robustez del sistema ante diferentes escenarios de calidad de datos. Las pruebas incluyeron:

\begin{itemize}
    \item \textbf{Datos críticos faltantes:} Simulación de ausencia de valores esenciales como pasos, frecuencia cardíaca y sueño.
    \item \textbf{Valores anómalos:} Inserción de datos fisiológicamente imposibles o extremos.
    \item \textbf{Datos parciales:} Combinación de valores válidos y faltantes para simular situaciones reales.
\end{itemize}

Los resultados mostraron un comportamiento excepcional del sistema:

\begin{itemize}
    \item \textbf{Precisión del 100\%:} Todas las alertas esperadas fueron generadas correctamente.
    \item \textbf{Consistencia:} El sistema mantuvo un comportamiento uniforme a través de diferentes usuarios y escenarios.
    \item \textbf{Priorización adecuada:} Las alertas se generaron con el nivel de prioridad correcto según la severidad.
\end{itemize}

\begin{table}[H]
\centering
\caption{Resultados del test de calidad de datos}
\begin{tabular}{|l|c|}
\hline
\textbf{Métrica} & \textbf{Valor} \\ \hline
Alertas totales evaluadas & 27 \\
Alertas esperadas y disparadas & 27 \\
Falsos positivos & 0 \\
Falsos negativos & 0 \\
Tasa de detección (Recall) & 100\% \\
Precisión (Precision) & 100\% \\
Tipos de alerta evaluados & 3 \\ \hline
\end{tabular}
\label{tab:resumen_calidad_datos}
\end{table}

Como se observa en la Tabla \ref{tab:resumen_calidad_datos}, el sistema demostró una capacidad excepcional para manejar datos de calidad variable, manteniendo una precisión perfecta en la detección de anomalías. Este resultado es particularmente relevante para el entorno real, donde los datos pueden ser incompletos o contener valores anómalos debido a fallos en los dispositivos o en la transmisión de datos.

\section{Test Avanzado: Combinación de Anomalías Clínicas}
\label{sec:test_combinadas}

Para validar la capacidad del sistema ante escenarios clínicos complejos, se diseñó un test en el que un mismo usuario presentaba múltiples anomalías el mismo día. Los datos insertados incluyeron:

\begin{itemize}
    \item \textbf{Sedentarismo extremo:} 900 minutos (↑125\%).
    \item \textbf{Disminución de sueño:} 300 minutos (↓28.5\%).
    \item \textbf{Frecuencia cardíaca anómala:} pico de 60 bpm con media de 73.8 y desviación estándar de 4.01.
    \item \textbf{Valor fisiológicamente incorrecto:} saturación de oxígeno en 18.5\%.
\end{itemize}

\noindent\textbf{Resultados esperados:}
\begin{itemize}
    \item Alerta por sedentarismo (\textit{high})
    \item Alerta por frecuencia cardíaca anómala (\textit{medium})
    \item Alerta por calidad de datos (\textit{high})
    \item Posible no activación de alerta de sueño si no supera el 30\%
\end{itemize}

\noindent\textbf{Resultados obtenidos:}
\begin{itemize}
    \item Todas las alertas esperadas fueron generadas correctamente.
    \item Se validó que las alertas múltiples se generan de forma independiente y no interfieren entre sí.
    \item Las prioridades asignadas fueron coherentes con la severidad de cada métrica.
\end{itemize}

Este test demuestra la madurez del sistema ante situaciones de salud comprometidas, donde múltiples factores deben ser considerados al mismo tiempo.

\subsection{Tolerancia a Datos Incompletos}
\label{subsec:datos_incompletos}

Durante las pruebas también se evaluó la respuesta del sistema ante datos incompletos, como la ausencia de valores de frecuencia cardíaca o saturación de oxígeno. Gracias a la regla específica de \textit{data quality}, el sistema generó alertas de forma precisa incluso en ausencia parcial de información fisiológica.

Este comportamiento es clave para mantener la fiabilidad en contextos reales, donde los dispositivos pueden fallar temporalmente o registrar datos incompletos por desconexiones o mal uso. Las alertas de calidad de datos no interfieren con otras alertas clínicas, pero permiten al personal médico evaluar la fiabilidad de los registros antes de tomar decisiones.

\section{Conclusiones}
\label{sec:conclusiones_validacion}

Los resultados de las pruebas confirman que el sistema es capaz de detectar con fiabilidad desviaciones críticas en los datos fisiológicos de los usuarios, tanto de forma individual como simultánea. Las reglas de umbral funcionan según lo diseñado, y el sistema gestiona adecuadamente tanto datos válidos como anómalos. Las áreas más sensibles, como la calidad de datos y la agregación de múltiples condiciones, se comportan correctamente, con una tasa de falsos positivos despreciable.

Los detalles completos, incluyendo los logs y configuraciones empleadas en cada prueba, se encuentran en el Anexo~\ref{anexo:pruebas}.
