% !TEX root = ../main.tex % Indica a algunos editores cuál es el fichero raíz
\chapter{Implementación}
\label{chap:implementacion}

Este capítulo detalla el proceso de construcción del sistema de monitorización remota, materializando el diseño arquitectónico expuesto en el Capítulo \ref{chap:diseno_arquitectura} en componentes software funcionales. Se aborda la configuración del entorno, la implementación de los componentes del backend (aplicación web y scripts de adquisición), la configuración y manejo de la base de datos, la lógica de procesamiento de datos, las medidas de seguridad aplicadas y la interfaz de visualización. Finalmente, se comentan algunos de los desafíos técnicos encontrados. El código fuente completo está disponible en el repositorio del proyecto \citep{github_repo_proyecto}.

\section{Entorno de Desarrollo y Tecnologías}
\label{sec:impl_entorno}

La Tabla~\ref{tab:tecnologias} resume las tecnologías principales empleadas en el desarrollo del sistema.

\begin{table}[htbp]
\caption{Tecnologías Principales del Sistema}
\label{tab:tecnologias}
\begin{tabular}{|p{0.2\textwidth}|p{0.7\textwidth}|}
\hline
\textbf{Categoría} & \textbf{Tecnologías} \\
\hline
Backend & Python 3.x, Flask, Flask-Login \\
\hline
Base de Datos & PostgreSQL + TimescaleDB, psycopg2 \\
\hline
Seguridad & cryptography (Fernet), OAuth 2.0 \\
\hline
Frontend & Bootstrap, Chart.js, Jinja2 \\
\hline
Automatización & cron, scripts Python \\
\hline
\end{tabular}
\end{table}

La elección de estas tecnologías permite una arquitectura modular, fácilmente mantenible y escalable, y facilita el despliegue en servidores Linux estándar. Todas las dependencias son de código abierto y ampliamente utilizadas en la industria, lo que garantiza soporte y seguridad a largo plazo.

\section{Implementación del Backend}
\label{sec:impl_backend}

El backend se estructura en dos componentes principales:

\subsection{Aplicación Web Flask}
La aplicación centraliza:
\begin{itemize}
    \item \textbf{Autenticación del Personal:} Gestiona el inicio y cierre de sesión del personal autorizado utilizando Flask-Login y credenciales almacenadas de forma segura (variables de entorno).
    \item \textbf{Gestión de Vinculaciones:} Proporciona rutas y plantillas HTML (Jinja2) para visualizar cuentas de Fitbit, asociar nombres a emails y gestionar vinculaciones activas.
    \item \textbf{Orquestación del Flujo OAuth 2.0:} Inicia el proceso de autorización con Fitbit, gestiona el intercambio de códigos por tokens mediante el módulo \texttt{auth.py}, y almacena los tokens cifrados en la base de datos (módulo \texttt{db.py}).
    \item \textbf{Precarga del Dashboard de Alertas:} Implementa una optimización que precarga en sesión los datos más recientes de cada usuario (resúmenes diarios, métricas intradía, registros de sueño y alertas) para acelerar la carga inicial del dashboard y mejorar la experiencia de usuario.
    \item \textbf{Visualización y Dashboard:} Sirve las vistas principales (dashboard de alertas, ficha de usuario, estadísticas) mediante rutas Flask y plantillas Jinja2, integrando Bootstrap y Chart.js para la visualización interactiva de datos clínicos y alertas.
\end{itemize}

\subsection{Scripts de Adquisición (\texttt{fitbit.py}, \texttt{fitbit\_intraday.py})}
Estos scripts operan de forma independiente, ejecutados por \texttt{cron}:
\begin{itemize}
    \item \textbf{Iteración sobre Usuarios:} Recuperan la lista de usuarios vinculados y sus credenciales cifradas de la base de datos. 
    \item \textbf{Gestión de Tokens:} Para cada usuario, descifran los tokens y gestionan su ciclo de vida: comprueban la expiración del token de acceso y, si es necesario, intentan refrescarlo utilizando el token de refresco y el endpoint correspondiente de la API de Fitbit. Los nuevos tokens se cifran y actualizan en la base de datos. Se implementa una gestión robusta de errores para evitar que un fallo de token detenga el procesamiento global.
    \item \textbf{Llamadas a la API de Fitbit:} Utilizan el token de acceso válido para solicitar los datos a la API de Fitbit, especificando el usuario, el periodo de tiempo y el formato deseado (JSON), utilizando la librería \texttt{requests}. Se implementa manejo de registro los errores para su posterior revisión.
    \item \textbf{Persistencia de Datos:} Una vez obtenidos y procesados los datos, los scripts llaman a las funciones de inserción del módulo \texttt{db.py} (ej. \texttt{insert\_intraday\_metric}, \texttt{insert\_daily\_summary}, \texttt{insert\_sleep\_log}) para almacenarlos en las hipertablas TimescaleDB correspondientes. La lógica de persistencia está desacoplada de la lógica de adquisición, facilitando el mantenimiento y la extensión futura.
    \item \textbf{Evaluación de Alertas:} Tras almacenar los datos, los scripts invocan la lógica de evaluación de alertas (módulo \texttt{alert\_rules.py}), que analiza los datos recientes y genera alertas clínicas si se cumplen los criterios definidos. Esta lógica está desacoplada y es fácilmente extensible.
\end{itemize}
La ejecución mediante \texttt{cron} asegura la recogida periódica y automatizada de datos sin intervención manual. La robustez frente a errores y la modularidad de los scripts son aspectos clave de la implementación.


\subsection{Criterios y Técnicas para la Generación de Alertas}
El sistema implementa un conjunto de reglas y umbrales (\textit{thresholds}) para la detección automática de situaciones de alerta en los datos de los usuarios. Estos umbrales han sido definidos combinando:

\begin{itemize}
    \item \textbf{Evidencia científica y guías clínicas:} Se han consultado estudios en gerontología, cardiología y medicina del sueño para establecer valores de referencia y cambios clínicamente relevantes~\cite{Smith2019, Owen2020, Irwin2015}.
    \item \textbf{Técnicas estadísticas:} Para la detección de anomalías en variables como la frecuencia cardíaca, se emplean métodos basados en la desviación estándar respecto a la media individual, lo que permite una personalización automática de los umbrales.
    \item \textbf{Porcentajes de cambio:} En métricas como pasos, tiempo sedentario o duración del sueño, se utilizan umbrales porcentuales (por ejemplo, caídas del 30\% o 50\%) para adaptarse a la línea base de cada usuario y detectar cambios significativos en su patrón habitual.
    \item \textbf{Validación empírica:} Los umbrales han sido ajustados y validados con datos de prueba para asegurar un equilibrio entre sensibilidad (detectar problemas reales) y especificidad (evitar falsas alarmas).
\end{itemize}

La lógica de generación de alertas incluye tanto reglas basadas en cambios diarios (porcentuales o absolutos) como la detección de anomalías intradía mediante análisis de series temporales. Para una descripción detallada de los umbrales, criterios y su justificación científica, véase la Tabla~\ref{tab:anexo_umbrales_alertas} en el anexo.

\subsection{Sistema de Internacionalización}
\label{subsec:impl_i18n}

El sistema implementa un robusto mecanismo de internacionalización (i18n) utilizando Flask-Babel y un sistema de traducciones estructurado:

\begin{itemize}
    \item \textbf{Gestión de Traducciones:} Las traducciones se organizan en el módulo \texttt{translations.py} mediante un diccionario jerárquico que cubre todas las áreas de la aplicación:
        \begin{itemize}
            \item Elementos comunes de la interfaz
            \item Mensajes de autenticación
            \item Textos del dashboard
            \item Gestión de dispositivos Fitbit
            \item Mensajes de error
            \item Sistema de alertas
        \end{itemize}
    \item \textbf{Integración con Flask:} Se utiliza Flask-Babel para:
        \begin{itemize}
            \item Detección automática del idioma del navegador
            \item Cambio dinámico de idioma durante la sesión
            \item Traducción de plantillas Jinja2
        \end{itemize}
    \item \textbf{Arquitectura Extensible:} El sistema está diseñado para facilitar la adición de nuevos idiomas y la actualización de traducciones existentes sin modificar el código base.
\end{itemize}

La implementación sigue las mejores prácticas de i18n, separando completamente el contenido de la lógica y permitiendo una experiencia de usuario consistente en múltiples idiomas.

\section{Implementación de la Persistencia (Base de Datos)}
\label{sec:impl_persistencia}

La capa de persistencia se basa en \textbf{PostgreSQL} extendido con \textbf{TimescaleDB}, gestionada a través del módulo \texttt{db.py} utilizando \texttt{psycopg2}. Esta combinación permite almacenar y consultar eficientemente grandes volúmenes de datos temporales y clínicos.

\subsection{Estructura de la Base de Datos}
El esquema principal incluye las siguientes tablas:
\begin{itemize}
    \item \textbf{users}: Información básica de los usuarios, tokens cifrados y metadatos de vinculación.
    \item \textbf{daily\_summaries}: Resúmenes diarios de actividad, sueño y biomarcadores por usuario y fecha (hipertabla TimescaleDB).
    \item \textbf{intraday\_metrics}: Métricas intradía (pasos, frecuencia cardíaca, calorías, minutos activos) con timestamp preciso (hipertabla TimescaleDB).
    \item \textbf{sleep\_logs}: Registros detallados de sueño por usuario y periodo (hipertabla TimescaleDB).
    \item \textbf{alerts}: Alertas clínicas generadas automáticamente, con tipo, prioridad, valores disparadores y estado de reconocimiento (hipertabla TimescaleDB).
\end{itemize}
Todas las tablas relevantes están vinculadas mediante claves foráneas y optimizadas con índices sobre los campos temporales y de usuario, lo que acelera las consultas y análisis longitudinales.

\subsection{Gestión y Acceso a Datos}
El módulo \texttt{db.py} abstrae todas las operaciones de acceso, inserción y actualización de datos, incluyendo:
\begin{itemize}
    \item \textbf{Conexión segura}: Uso de credenciales gestionadas por variables de entorno.
    \item \textbf{Gestión de tokens}: Almacenamiento cifrado de tokens OAuth 2.0 mediante la librería \texttt{cryptography}.
    \item \textbf{Inserción eficiente}: Funciones como \texttt{insert\_daily\_summary}, \texttt{insert\_intraday\_metric}, \texttt{insert\_sleep\_log} y \texttt{insert\_alert} utilizan inserciones masivas y gestionan conflictos con \texttt{ON CONFLICT DO UPDATE} para evitar duplicados y mantener la integridad.
    \item \textbf{Consultas optimizadas}: Funciones para recuperar datos diarios, intradía, de sueño y alertas, filtrando por usuario y rango temporal, y devolviendo los resultados en formatos útiles para el backend y la visualización.
\end{itemize}


\subsection{Optimización de Consultas con TimescaleDB}
\label{subsec:impl_timescaledb_optimization}

La implementación utiliza TimescaleDB como extensión de PostgreSQL para gestionar eficientemente los datos de series temporales. El sistema implementa las siguientes estrategias:

\begin{itemize}
    \item \textbf{Estructura de Datos:} Se utilizan tablas específicas para cada tipo de dato temporal (\texttt{daily\_summaries}, \texttt{intraday\_metrics}, \texttt{sleep\_logs}), con índices optimizados sobre los campos de tiempo y usuario.
    \item \textbf{Consultas Eficientes:} Las consultas están diseñadas para ser simples y directas, filtrando por usuario y rangos temporales específicos, lo que permite aprovechar los índices de la base de datos.
    \item \textbf{Gestión de Conexiones:} Se implementa un gestor de base de datos (\texttt{DatabaseManager}) que maneja eficientemente las conexiones y transacciones, asegurando un uso óptimo de los recursos.
\end{itemize}

Los detalles de implementación, incluyendo las consultas SQL y la estructura de las tablas, se encuentran en el Anexo~\ref{annex:code:db_implementation}. 

\subsection{Robustez y Manejo de Errores}
\label{subsec:robustez_errores}

El sistema ha sido diseñado para ser robusto y tolerante a fallos, asegurando la continuidad del servicio incluso ante incidencias en la adquisición de datos, errores de la API de Fitbit o problemas en la base de datos. A continuación se describen los principales mecanismos implementados:

\begin{itemize}
    \item \textbf{Gestión de errores en la adquisición de datos:} Los scripts de adquisición (\texttt{fitbit.py}, \texttt{fitbit\_intraday.py}) encapsulan cada petición a la API de Fitbit en bloques \texttt{try...except}, registrando los errores y continuando con el siguiente usuario en caso de fallo. Esto evita que un error puntual detenga la recolección global de datos.
    \item \textbf{Manejo de tokens expirados o inválidos:} Si se detecta un token de acceso expirado, el sistema intenta refrescarlo automáticamente utilizando el token de refresco. Si el refresco falla, se registra el incidente y se notifica la necesidad de reautorización, sin afectar al resto de usuarios.
    \item \textbf{Validación y limpieza de datos:} Antes de almacenar los datos, se validan los formatos y rangos fisiológicos esperados. Los valores nulos, inconsistentes o fuera de rango se gestionan adecuadamente para evitar la corrupción de la base de datos.
    \item \textbf{Persistencia atómica y control de integridad:} Las operaciones de inserción y actualización en la base de datos utilizan transacciones atómicas y mecanismos de \texttt{ON CONFLICT DO UPDATE} para evitar duplicados y mantener la integridad referencial.
    \item \textbf{Logging detallado:} Todos los errores y eventos relevantes se registran en archivos de log, facilitando la monitorización y el diagnóstico de incidencias.
    \item \textbf{Diseño modular y desacoplado:} La separación entre adquisición, procesamiento, almacenamiento y generación de alertas permite aislar fallos y facilita la recuperación ante errores, así como la extensión futura del sistema.
\end{itemize}

Estos mecanismos aseguran que el sistema pueda operar de forma continua y fiable, minimizando el impacto de errores puntuales y facilitando el mantenimiento y la escalabilidad a largo plazo.

\section{Implementación de la Seguridad (OAuth, RGPD)}
\label{sec:impl_seguridad}

La seguridad y el cumplimiento normativo (RGPD \citep{rgpd_texto_oficial}) fueron consideraciones centrales durante la implementación.

La integración con Fitbit se implementó siguiendo las mejores prácticas de OAuth 2.0 \citep{oauth_spec_rfc6749}:
\begin{itemize}
    \item \textbf{Flujo Authorization Code con PKCE:} Se implementó este flujo, considerado el más seguro para aplicaciones web con backend \citep{oauth_security_bcp_rfc8252}. El módulo \texttt{auth.py} y las rutas de Flask en \texttt{app.py} gestionan la generación del \texttt{code\_verifier} y \texttt{code\_challenge}, el parámetro \texttt{state} para prevenir CSRF, el intercambio del código de autorización por tokens, y el manejo seguro de \texttt{client\_id} y \texttt{client\_secret}.
    \item \textbf{Gestión Segura de Tokens:} Los tokens de acceso y refresco se consideran información altamente sensible. Se cifran inmediatamente después de su obtención utilizando cifrado simétrico (AES mediante Fernet en la librería \texttt{cryptography}) con una clave secreta gestionada externamente (variable de entorno, no en el código fuente). Solo se descifran en memoria en el momento exacto de su uso.
    \item \textbf{Refresco de Tokens:} La lógica para refrescar tokens caducados se implementó de forma robusta, manejando posibles errores y actualizando los tokens en la base de datos de forma atómica.
    \item \textbf{HTTPS:} Aunque la configuración de HTTPS es a nivel de despliegue (servidor web/proxy inverso), el diseño asume y requiere que toda la comunicación (frontend-backend, backend-Fitbit API) se realice sobre HTTPS para proteger los datos en tránsito. Se recomienda seguir buenas prácticas de seguridad web como las delineadas por OWASP \citep{owasp_top10}.
\end{itemize}
Los detalles del flujo OAuth 2.0 y la gestión de tokens se encuentran en el Anexo \ref{anexo:oauth_fitbit}.

\section{Implementación de la Visualización}
\label{sec:impl_visualizacion}

La interfaz de visualización se implementa utilizando Flask como backend y una combinación de Bootstrap 5 y Chart.js para el frontend. La arquitectura se estructura en tres niveles:
\begin{table}[htbp]
    \centering
    \begin{tabular}{|l|l|}
        \hline
        \textbf{Capa} & \textbf{Componentes} \\
        \hline
        Backend (Flask) & • Rutas protegidas con \texttt{@login\_required} y soporte multilingüe \\
        & • Procesamiento y filtrado de datos en servidor \\
        & • Gestión de sesiones para optimización \\
        & • APIs REST para datos dinámicos \\
        \hline
        Frontend & • Bootstrap 5 para diseño responsivo \\
        & • Chart.js para gráficos interactivos \\
        & • JavaScript para interacciones dinámicas \\
        & • FontAwesome e iconos Bootstrap para UI \\
        \hline
        Integración & • Plantillas Jinja2 para renderizado \\
        & • AJAX para actualizaciones dinámicas \\
        & • Caché basado en sesiones Flask \\
        & • Exportación de datos a CSV \\
        \hline
    \end{tabular}
    \caption{Arquitectura de la visualización por capas}
    \label{tab:arquitectura_visualizacion}
\end{table}

\subsection{Layout del Dashboard de Alertas}
\label{subsec:layout_dashboard}

El dashboard de alertas se implementa como una vista Flask que integra Bootstrap 5 y Chart.js, estructurada en:

\begin{itemize}
    \item Panel superior con contadores de alertas por prioridad y estado
    \item Filtros avanzados (fecha, usuario, tipo, prioridad, estado)
    \item Tabla de alertas con indicadores visuales y datos intradía relevantes
    \item Funcionalidades de exportación y soporte multilingüe
\end{itemize}

\subsection{Interactividad y Actualización}
\label{subsec:interactividad}

La interactividad del dashboard se implementa mediante una combinación de rutas Flask y llamadas AJAX:

\begin{itemize}
    \item \textbf{Filtrado Dinámico:} Los filtros del dashboard se procesan en el servidor mediante la ruta \texttt{/livelyageing/dashboard/alerts}, que construye consultas SQL dinámicas basadas en los parámetros recibidos.
    \item \textbf{Reconocimiento de Alertas:} Se implementa mediante llamadas AJAX a la ruta \texttt{/livelyageing/api/alerts/<alert\_id>/acknowledge}, permitiendo actualizar el estado de las alertas sin recargar la página.
    \item \textbf{Visualización de Detalles:} Al hacer clic en una alerta, se cargan los datos intradía relevantes (frecuencia cardíaca, pasos, etc.) mediante una llamada AJAX a \texttt{/livelyageing/api/alerts/<alert\_id>}, mostrándolos en un modal con gráficos Chart.js.
    \item \textbf{Optimizaciones:}
        \begin{itemize}
            \item Precarga de datos en sesión para mejorar el tiempo de respuesta inicial
            \item Paginación del lado del servidor para manejar grandes volúmenes de alertas
        \end{itemize}
\end{itemize}

\subsection{Implementación del Dashboard de Usuarios y Ficha de Usuario}
\label{subsec:impl_dashboard_usuarios}

La interfaz de usuarios tiene dos componentes principales:

\begin{itemize}
    \item \textbf{Dashboard de Usuarios (\texttt{/livelyageing/home}):}
        \begin{itemize}
            \item Listado de usuarios vinculados con estado de actividad
            \item Búsqueda y filtrado por nombre o email
            \item Acciones rápidas de gestión (reasignación, desvinculación)
            \item Indicadores visuales de estado de sincronización
        \end{itemize}
    \item \textbf{Ficha de Usuario (\texttt{/livelyageing/user/<user\_id>}):}
        \begin{itemize}
            \item Vista detallada con múltiples pestañas de información
            \item Gráficos interactivos de series temporales (Chart.js)
            \item Resumen de métricas clave del día
            \item Gráficos de intervalos de inactividad
            \item Listado de alertas recientes con contexto
        \end{itemize}
\end{itemize}

\section{Desafíos y Soluciones Técnicas}
\label{sec:impl_desafios}

Durante la implementación se abordaron varios desafíos técnicos principales:

\begin{itemize}
    \item \textbf{Gestión de Tokens OAuth 2.0:} Implementación del ciclo de autenticación Fitbit con cifrado de tokens, refresco automático y persistencia durante reasignaciones.
    
    \item \textbf{Gestión de Múltiples Dispositivos:} Sistema robusto de vinculación y reasignación con selección forzada de cuenta, preservación de históricos y soporte multilingüe.
    
    \item \textbf{Sistema de Alertas:} Implementación de un sistema de detección de anomalías basado en evidencia científica, con umbrales personalizados y validación rigurosa.

\end{itemize}

Estos desafíos técnicos fueron abordados mediante soluciones robustas y escalables, resultando en un sistema fiable y eficiente que cumple con los requisitos establecidos.

